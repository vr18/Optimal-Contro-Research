\documentclass{article}
\begin{document}

\begin{center}

{\bf RESEARCH BASICS}

\end{center} 

Generally, there is a vibration in structures like bridges, buildings etc. due to the transfer of energy from fluids to the structures. These forces are also known as self-excited or aero-elastic forces or the phenomenon is also known as flutter.\\
These kinds of forces generally have a negative effect on the bodies. But, flutter can generally be used positively. For Eg. the energy from the vibration of bodies can be converted to electrical energy. Not a lot of research has been carried out in this field.\\
One group is trying to find out flutter derivatives for rectangular deck. My task will be to convert the energy from the vibrating structures into electrical energy.\\
When the structure comes in contact with the fluid for the first time, energy gets transfered from the fluid to the structure after which continuous contact results in self-excitation of the structure. Energy can generally be extracted from a body in this mode by connecting a generator (may be linear in this case.\\
According to the theory, the generator exerts a force on the structure. This force restricts the motion of the structure. This restricting or negative force facilitates energy transfer.\\
The Equation of Motion for a 1-DOF structure can be written as:\\
$m\ddot{y} + c\dot{y} + ky = f_{body} + u$\\
Here, u $\rightarrow$ control force provided by the PTO(Power Take off Unit) Unit.\\
c is the radiation damping component. Here, radiation means the effect of the PTO downstream of the wind.\\

Now, the velocity of vibration of the structure can be represented as $\dot{y(t)}$, i.e. it is the y-component of velocity.\\
Hence, Power = u$\dot{y(t)}$\\
Similarly, Energy, E = Power * time $\Rightarrow \int^T_0u\dot{y(t)}\,dt$\\
From the above equation, we are supposed to find u(t) that maximizes E. We do this because, fluid-structure interaction does not remain the same continuously over a period of time but instead varies.\\
This type of problem is known as Optimal Control Problem.\\
Subject to constraints\\
$y_(min) < y < y_(max)$\\  
$\dot{y_(min)} < \dot{y} < \dot{y_(max)}$\\
$u_(min) < u < u_(max)$\\


Reviewing the paper {\bf Optimal Control of Wave Energy Converters}\\
\end{document}
